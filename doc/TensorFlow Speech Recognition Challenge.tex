%----------------------------------------------------------------------------------------
%	PACKAGES AND OTHER DOCUMENT CONFIGURATIONS
%----------------------------------------------------------------------------------------

\documentclass[twoside]{article}

\usepackage[sc]{mathpazo} % Use the Palatino font
\usepackage[T1]{fontenc} % Use 8-bit encoding that has 256 glyphs
\linespread{1.05} % Line spacing - Palatino needs more space between lines
\usepackage{microtype} % Slightly tweak font spacing for aesthetics

\usepackage[english]{babel} % Language hyphenation and typographical rules

\usepackage[hmarginratio=1:1,top=32mm,columnsep=20pt]{geometry} % Document margins
\usepackage[hang, small,labelfont=bf,up,textfont=it,up]{caption} % Custom captions under/above floats in tables or figures
\usepackage{booktabs} % Horizontal rules in tables

\usepackage{lettrine} % The lettrine is the first enlarged letter at the beginning of the text

\usepackage{enumitem} % Customized lists
\setlist[itemize]{noitemsep} % Make itemize lists more compact

\usepackage{titlesec} % Allows customization of titles
\renewcommand\thesection{\Roman{section}} % Roman numerals for the sections
\renewcommand\thesubsection{\roman{subsection}} % roman numerals for subsections
\titleformat{\section}[block]{\large\scshape\centering}{\thesection.}{1em}{} % Change the look of the section titles
\titleformat{\subsection}[block]{\large}{\thesubsection.}{1em}{} % Change the look of the section titles

\usepackage{fancyhdr} % Headers and footers
\pagestyle{fancy} % All pages have headers and footers
\fancyhead{} % Blank out the default header
\fancyfoot{} % Blank out the default footer
\fancyhead[C]{TensorFlow Speech Recognition Challenge $\bullet$ May 2018} % Custom header text
\fancyfoot[RO,LE]{\thepage} % Custom footer text

\usepackage{titling} % Customizing the title section

\usepackage{hyperref} % For hyperlinks in the PDF

\usepackage{multicol}
\usepackage{graphicx}
\graphicspath{{img/}}
\usepackage{float}
\renewcommand{\figurename}{Fig.}

\usepackage{amssymb,amsmath,amsthm}
\newtheorem{theorem}{Theorem}
\newtheorem{defn}{Definition}

\newcommand{\code}[1]{\texttt{#1}}

\newenvironment{Figure}
  {\par\medskip\noindent\minipage{\linewidth}}
  {\endminipage\par\medskip}

%----------------------------------------------------------------------------------------
%	TITLE SECTION
%----------------------------------------------------------------------------------------

\setlength{\droptitle}{-4\baselineskip} % Move the title up

\pretitle{\begin{center}\Huge\bfseries} % Article title formatting
\posttitle{\end{center}} % Article title closing formatting
\title{TensorFlow Speech Recognition Challenge} % Article title

\author{%
\textsc{Petra Br\v{c}i\'c}\\% \\[1ex] 
\normalsize Applied Mathematics \\
\normalsize \href{mailto:petrabrcic94@gmail.com}{petrabrcic94@gmail.com} 
\and
\textsc{Ivan \v{C}eh}\\% \\[1ex] 
\normalsize Computer Science \\
\normalsize \href{mailto:ivan.ceh1234@gmail.com}{ivan.ceh1234@gmail.com} 
\and
\textsc{Sandro Lovni\v{c}ki}\\% \\[1ex] 
\normalsize Computer Science \\ 
\normalsize \href{mailto:lovnicki.sandro@gmail.com}{lovnicki.sandro@gmail.com}
}

\date{\today} 
\renewcommand{\maketitlehookd}{%
\noindent \textbf{\\Abstract:} We are witnessing the significant development of the blooming relationship between science and technology. Humanity is striving on making everyday life easier and more controllable by the integration of the intelligent machines. We want to control those machine easier, such as by spoken commands. This is where this project comes in and answers that question.
\\\\
\textbf{Keywords:} tensorflow, speech recognition, audio recognition, feature extraction, MFCC.
}


%----------------------------------------------------------------------------------------

\begin{document}

% Print the title and table of contents
\maketitle
\tableofcontents
\newpage

%----------------------------------------------------------------------------------------
%	ARTICLE CONTENTS
%----------------------------------------------------------------------------------------
\begin{multicols}{2}

\section{Introduction}

\ldots
for which our code can be found at our \hyperref[https://github.com/Qkvad/3rdOrderTensors]{GitHub repository}\footnote{https://github.com/Qkvad/SpeechRecognition}.

%------------------------------------------------
\section{Data}
This project was given by Kaggle as Tensorflow challenge, where the task is to correctly classify the audio recording. Dataset was given by Tensorflow as one second audio clips of different words spoken by different people. Dataset can automaticly be downloaded when starting the train part of machine learning solution to the problem. 
The goal is to have a model that tries to classify a one second audio clip as either silence, an unknown word, "yes", "no", "up", "down", "left", "right", "on", "off", "stop" and "go", even though more words can be found in the dataset and included in the training process. Each of the words is separated into files named as the recordings of the word they present. In order to make these clips as realistic as they would be in real life, the background noise is added in preprocess part of the code. There is a file of few background noises within the dataset that is downloaded. 
Each audio file is encoded with the id of the person who recorded the word, followed by nohash and a number from 0, specifing the time the same person recorded the word.  
Apart from the audio files that take up the majority of the data file, testing and validation lists can be found too. Each of those text file contains a list of the audio clips so that dataset can be patitioned in training, validation and test set. 
\subsection{Preprocessing}
In order to feed our neural network with the data, we first need to process it. Neural network is fed by the image, so we need to tranform the audio file to a spectrogram. Spectrogram is a visual representation of the spectrum of frequencies of sound or other signal as they vary in time. Spectrograms can be generated by an optical spectrometer, a bank of band-pass filters or by Fourier transorm. 
First, let us observe what the data looks like after reading the audio file. \\
 
\begin{Figure}
	\centering
	\includegraphics[width=0.8\textwidth]{happy-read}
	\captionof{figure}{Waveform - Sound preasure / Time(ms)}
	\label{img:pocetnoStanje}
\end{Figure}
%------------------------------------------------
\section{Machine Learning}

\subsection{Convolutional Neural Network}

\subsection{Results}

%------------------------------------------------
\section{Different Approaches}


%------------------------------------------------


%----------------------------------------------------------------------------------------
%	REFERENCE LIST
%----------------------------------------------------------------------------------------

\begin{thebibliography}{99}

\bibitem[1]{kilmer-braman-hao}
Misha E. Kilmer, Karen Braman, Ning Hao.
\newblock Third Order Tensors as Operators on Matrices: A
Theoretical and Computational Framework with
Applications in Imaging

\bibitem[2]{kilmer-martin}
Misha E. Kilmer, Carla D. Martin.
\newblock Factorization Strategies for Third-order Tensors

 
\end{thebibliography}

%----------------------------------------------------------------------------------------

\end{multicols}{2}
\end{document}
